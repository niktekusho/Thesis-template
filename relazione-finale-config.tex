%**************************************************************
% file contenente le impostazioni della tesi
%**************************************************************

%**************************************************************
% Frontespizio
%**************************************************************

% Autore
\newcommand{\myName}{Nicola Dal Maso}
\newcommand{\myTitle}{Un prototipo di sistema di Home Automation basato su microservizi}

% Tipo di tesi
\newcommand{\myDegree}{Tesi di laurea triennale}

% Università
\newcommand{\myUni}{Università degli Studi di Padova}

% Facoltà
\newcommand{\myFaculty}{Corso di Laurea in Informatica}

% Dipartimento
\newcommand{\myDepartment}{Dipartimento di Matematica "Tullio Levi-Civita"}

% Titolo del relatore
\newcommand{\profTitle}{Prof. }

% Relatore
\newcommand{\myProf}{Tullio Vardanega}

% Luogo
\newcommand{\myLocation}{Padova}

% Anno accademico
\newcommand{\myAA}{2017-2018}

% Data discussione
\newcommand{\myTime}{Febbraio 2018}

% checkmark
\newcommand{\cmark}{\ding{51}}%

% non checkmark
\newcommand{\xmark}{\ding{55}}%


%**************************************************************
% Impostazioni di impaginazione
% see: http://wwwcdf.pd.infn.it/AppuntiLinux/a2547.htm
%**************************************************************

\setlength{\parindent}{14pt}   % larghezza rientro della prima riga
\setlength{\parskip}{0pt}   % distanza tra i paragrafi


%**************************************************************
% Impostazioni di biblatex
%**************************************************************
\bibliography{bibliografia} % database di biblatex

\defbibheading{bibliography} {
    \cleardoublepage
    \phantomsection
    \addcontentsline{toc}{chapter}{\bibname}
    \chapter*{\bibname\markboth{\bibname}{\bibname}}
}

\setlength\bibitemsep{1.5\itemsep} % spazio tra entry

\DeclareBibliographyCategory{opere}
\DeclareBibliographyCategory{web}

\addtocategory{opere}{womak:lean-thinking}
\addtocategory{web}{site:agile-manifesto}

\defbibheading{opere}{\section*{Riferimenti bibliografici}}
\defbibheading{web}{\section*{Siti Web consultati}}


%**************************************************************
% Impostazioni di caption
%**************************************************************
\captionsetup{
    tableposition=top,
    figureposition=bottom,
    font=small,
    format=hang,
    labelfont=bf
}

%**************************************************************
% Impostazioni di glossaries
%**************************************************************
\renewcommand{\glstextformat}[1]{#1\glsfirstoccur}

%**************************************************************
% Acronimi
%**************************************************************
\renewcommand{\acronymname}{Acronimi e abbreviazioni}

\newacronym[description={\glslink{apig}{Application Program Interface}}]
    {api}{API}{Application Program Interface}

\newacronym[description={\glslink{umlg}{Unified Modeling Language}}]
    {uml}{UML}{Unified Modeling Language}

\newacronym{mit-acronym}{MIT}{Massachusetts Institute of Technology}

\newacronym{cli-acronym}{CLI}{Command Line Interface}

\newacronym{cpu-acronym}{CPU}{Central Processing Unit}

\newacronym{gpu-acronym}{GPU}{Graphics Processing Unit}

\newacronym{ide-acronym}{IDE}{Integrated Development Environment}

\newacronym{dom-acronym}{DOM}{Document Object Model}

\newacronym{qos-acronym}{QoS}{Quality of Service}

\newacronym{mqtt-acronym}{MQTT}{MQ Telemetry Transport}

%**************************************************************
% Glossario
%**************************************************************
\renewcommand{\glossaryname}{Glossario}

\newglossaryentry{apig}
{
    name=\glslink{api}{API},
    text=\emph{Application Program Interface},
    sort=api,
    description={in informatica con il termine \emph{Application Programming Interface API} (ing. interfaccia di programmazione di un'applicazione) si indica ogni insieme di procedure disponibili al programmatore, di solito raggruppate a formare un set di strumenti specifici per l'espletamento di un determinato compito all'interno di un certo programma. La finalità è ottenere un'astrazione, di solito tra l'hardware e il programmatore o tra software a basso e quello ad alto livello semplificando così il lavoro di programmazione}
}

\newglossaryentry{umlg}
{
    name=\glslink{uml}{UML},
    text=UML,
    sort=uml,
    description={in ingegneria del software \emph{UML, Unified Modeling Language} (ing. linguaggio di modellazione unificato) è un linguaggio di modellazione e specifica basato sul paradigma object-oriented. L'\emph{UML} svolge un'importantissima funzione di ``lingua franca'' nella comunità della progettazione e programmazione a oggetti. Gran parte della letteratura di settore usa tale linguaggio per descrivere soluzioni analitiche e progettuali in modo sintetico e comprensibile a un vasto pubblico}
}

\newglossaryentry{load balancerg}
{
  name=\emph{Load balancer},
  text=load balancer,
  sort=load balancer,
  description={in informatica, il load balancer è lo strumento hardware o software attraverso cui viene distribuito un carico di lavoro su più risorse di elaborazione (\textit{load balancing})}
}

\newglossaryentry{throughputg}
{
  name=\emph{Throughput},
  text=throughput,
  sort=throughput,
  description={indica la capacità di un canale di comunicazione di processare o trasmettere dati in uno specifico periodo di tempo. É una misura di produttività.}
}

\newglossaryentry{aspettativag}
{
  name=Aspettativa,
  text=aspettativa,
  sort=aspettativa,
  description={è un periodo di astensione dal lavoro, previsto dalla legge, che il datore di lavoro può concedere ad un proprio lavoratore per motivi familiari o personali, generalmente non retribuito}
}

\newglossaryentry{kernelg}
{
  name=\emph{Kernel},
  text={Kernel},
  sort=kernel,
  description={è il sottosistema di un sistema operativo che fornisce ai processi in esecuzione sull'elaboratore l'accesso alle risorse fisiche, in modo controllato e sicuro}
}

\newglossaryentry{MIT}
{
  name=\gls{mit-acronym},
  text=MIT,
  sort=mit,
  description={nell'ambito delle licenze \emph{software}, la licenza MIT è una licenza creata dal \emph{Massachusetts Institute of Technology} per il rilascio di codice sorgente secondo condizioni permissive da parte dell'utilizzatore del codice, tutelando il \emph{copyright} dell'autore}
}

\newglossaryentry{broadcast}
{
  name=\emph{Broadcast},
  text={Broadcast},
  sort=broadcast,
  description={nell'ambito delle telecomunicazione, indica una trasmissione delle informazioni da un sistema trasmittente ad un insieme di apparati riceventi non conosciuto a priori (uno a molti)}
}

\newglossaryentry{pattern}
{
  name=\emph{Pattern},
  text={Pattern},
  sort=pattern,
  description={in informatica è una soluzione testata e dimostrata ad un problema ricorrente. La sua accezione principale consiste in quella di \emph{Design Pattern} (soluzioni progettuali)}
}

\newglossaryentry{Gatewaypattern}
{
  name=\emph{Gateway Pattern},
  text={Gateway Pattern},
  sort=gateway pattern,
  description={è un \gls{pattern} che propone una soluzione al problema "Come un \emph{client} può interagire con un sistema a microservizi?". Questo è un peculiare problema delle architetture a microservizi e la soluzione proposta consiste nel sviluppare un servizio la cui funzionalità principale consiste nel gestire le richieste dei \emph{client} in un singolo \emph{entrypoint} pubblico, che si occupa di redirezionare la richiesta al servizio adeguato o assembla una risposta complessa a partire dalle informazioni messe a disposizione da più servizi\footcite{gateway-pattern}}
}

\newglossaryentry{js}
{
  name=\emph{JavaScript},
  text={JavaScript},
  sort={javascript},
  description={è un linguaggio di programmazione nato per aggiungere dinamicità e interattività alle pagine renderizzate dai browser attraverso l'esecuzione di semplici istruzioni che manipolino la struttura della pagina \emph{web} visualizzata}
}

\newglossaryentry{broker}
{
  name=\emph{Broker},
  text={broker},
  sort={broker},
  description={un broker nell'ambito MQTT è un server che reindirizza i messaggi pubblicati verso i client sottoscritti all'argomento di ciascun messaggio}
}

\newglossaryentry{milestone}
{
  name=\emph{Milestone},
  text={milestone},
  sort={milestone},
  description={nell'ambito della pianificazione di un progetto, indica il raggiungimento di obiettivi stabiliti in fase di definizione del progetto}
}

\newglossaryentry{open source}
{
  name=\emph{Open source},
  text={open source},
  sort={open source},
  description={indica un prodotto \emph{software} in cui il codice sorgente è pubblicato dagli autori al fine di favorirne lo studio, aumentarne la diffusione e permettere alla comunità di apportarvi modifiche}
}

\newglossaryentry{CLI}
{
  name=\gls{cli-acronym},
  text=CLI,
  sort=cli,
  description={indica un'interfaccia utente in cui l'interazione avviene attraverso la digitazione di comandi testuali}
}

\newglossaryentry{CPU}
{
  name=\gls{cpu-acronym},
  text=CPU,
  sort=cpu,
  description={è un tipo di dispositivo \emph{hardware} dedicato all'esecuzione di istruzioni generiche definite in un insieme di istruzioni eseguibili (\emph{instruction set})}
}

\newglossaryentry{GPU}
{
  name=\gls{gpu-acronym},
  text=GPU,
  sort=gpu,
  description={è un tipo di dispositivo \emph{hardware} specializzato nell'esecuzione di istruzioni per la visualizzazione delle immagini a schermo}
}

\newglossaryentry{mocking}
{
  name=\emph{Mock},
  text=mock,
  sort=mock,
  description={nell'ambito della programmazione ad oggetti, indica un oggetto che simula il comportamento di un altro oggetto non accessibile o non implementato. I \emph{mock} sono utilizzati principalmente per isolare un oggetto dalle sue dipendenze durante le attività di test}
}

\newglossaryentry{IDE}
{
  name=\gls{ide-acronym},
  text=IDE,
  sort=ide,
  description={è un \emph{software} che aiuta gli sviluppatori nella scrittura del codice sorgente di un programma, segnalando errori di sintassi e fornendo funzionalità per l'ispezione del codice alla ricerca di errori}
}

\newglossaryentry{Google}
{
  name=Google,
  text=Google,
  sort=google,
  description={è un'azienda statunitense che opera nel campo dei servizi \emph{online} e della telefonia (attraverso il sistema operativo Android)}
}

\newglossaryentry{Microsoft}
{
  name=Microsoft,
  text=Microsoft,
  sort=microsoft,
  description={è un'azienda statunitense che opera nel campo dell'informatica, sviluppando sistemi operativi, \emph{software} per la produttività aziendale e domestica ed elettronica di consumo}
}

\newglossaryentry{GitHub}
{
  name=GitHub,
  text=GitHub,
  sort=github,
  description={è un servizio che permette di mantenere documenti e progetti \emph{software} all'interno di un controllo di versione distribuito. Il servizio è sviluppato dall'omonima azienda}
}

\newglossaryentry{DOM}
{
  name=\gls{dom-acronym},
  text=DOM,
  sort=dom,
  description={è una forma di rappresentazione per documenti strutturati utilizzando modelli orientati agli oggetti}
}

\newglossaryentry{CPU-bound}
{
  name=\emph{CPU-bound},
  text=CPU-bound,
  sort=cpu-bound,
  description={l'espressione indica quei processi che sfruttano le risorse di elaborazione di una CPU, ma non richiedono servizi di scambio dati con l'esterno. Un esempio di \emph{software} tipicamente \emph{CPU-bound} riguarda i programmi di calcolo matematico}
}

\newglossaryentry{overhead}
{
  name=\emph{Overhead},
  text=overhead,
  sort=overhead,
  description={indica le risorse richieste in più rispetto a quelle necessarie per ottenere un determinato scopo}
}

\newglossaryentry{QoS}
{
  name=\gls{qos-acronym},
  text=QoS,
  sort=qos,
  description={nell'ambito delle reti di telecomunicazioni, il QoS indica i parametri usati per qualificare le prestazioni e l'affidabilità di un servizio offerto dalla rete}
}

\newglossaryentry{handshake}
{
  name=\emph{Handshake},
  text=handshake,
  sort=handshake,
  description={in informatica rappresenta il processo attraverso cui due elaboratori stabiliscono un insieme di regole comuni per la comunicazione di dati}
}

\newglossaryentry{MQTT}
{
  name=\gls{mqtt-acronym},
  text=MQTT,
  sort=mqtt,
  description={è un protocollo di comunicazione leggero basato sullo scambio di messaggi tra dispositivi caratterizzati da risorse di elaborazione limitate in una rete inaffidabile}
}

\newglossaryentry{abstract-factory}
{
  name=\emph{Abstract Factory},
  text=Abstract Factory,
  sort=abstract factory,
  description={è un \gls{pattern} creazionale che fornisce un'interfaccia per creare famiglie di oggetti connessi o dipendenti tra loro, in modo che non ci sia necessità da parte dei \emph{client} di specificare quale classe istanziare. Questo pattern permette che un sistema sia indipendente dall'implementazione degli oggetti concreti e che il \emph{client}, attraverso l'interfaccia, utilizzi diverse famiglie di prodotti}
}

\newglossaryentry{WebSocket}
{
  name=\emph{WebSocket},
  text=WebSocket,
  sort=websocket,
  description={è una tecnologia che fornisce canali di comunicazione in cui le informazioni viaggiano sia dal \emph{client} al server che viceversa e permette di realizzare applicazioni \emph{web} in cui vi è comunicazione in tempo reale tra il server e il \emph{client}}
}

\newglossaryentry{dp-composite}
{
  name=\emph{Composite},
  text=Composite,
  sort=composite,
  description={è un \gls{pattern} strutturale che permette di effettuare operazioni su gruppi di oggetti come se essi fossero l'istanza di un singolo oggetto e permette di manipolare oggetti singoli e loro composizioni in modo uniforme}
}

\newglossaryentry{repository}
{
  name=\emph{Repository},
  text=repository,
  sort=repository,
  description={è un sistema \emph{software} che archivia un insieme di \emph{file} e cartelle e le relative informazioni al fine di permettere la distribuzione sicura e affidabile di questo insieme di dati}
}

\newglossaryentry{sharding}
{
  name=\emph{Sharding},
  text=sharding,
  sort=sharding,
  description={nell'ambito delle tecnologie per i \emph{database}, è il procedimento attraverso cui un \emph{database} principale partiziona i dati ricevuti su molteplici istanze del \emph{database}, definite \emph{slave}. I dati in una architettura di questo tipo sono frammentati tra \emph{database} separati, potenzialmente aumentando le prestazioni e la resilienza dei dati}
}

\newglossaryentry{callback}
{
  name=\emph{Callback},
  text=Callback,
  sort=callback,
  description={è un \gls{pattern} che permette di delegare l'esecuzione di una funzione, passandola in \emph{input} di un'altra funzione. Le funzioni di \emph{callback} sono invocate per propagare i risultati di un'operazione complessa attraverso diversi \emph{step}.}
}
 % database di termini
\makeglossaries


%**************************************************************
% Impostazioni di graphicx
%**************************************************************
\graphicspath{{immagini/}} % cartella dove sono riposte le immagini


%**************************************************************
% Impostazioni di hyperref
%**************************************************************
\hypersetup{
    %hyperfootnotes=false,
    %pdfpagelabels,
    %draft,	% = elimina tutti i link (utile per stampe in bianco e nero)
    colorlinks=true,
    linktocpage=true,
    pdfstartpage=1,
    pdfstartview=FitV,
    % decommenta la riga seguente per avere link in nero (per esempio per la stampa in bianco e nero)
    %colorlinks=false, linktocpage=false, pdfborder={0 0 0}, pdfstartpage=1, pdfstartview=FitV,
    breaklinks=true,
    pdfpagemode=UseNone,
    pageanchor=true,
    pdfpagemode=UseOutlines,
    plainpages=false,
    bookmarksnumbered,
    bookmarksopen=true,
    bookmarksopenlevel=1,
    hypertexnames=true,
    pdfhighlight=/O,
    %nesting=true,
    %frenchlinks,
    urlcolor=webbrown,
    linkcolor=RoyalBlue,
    citecolor=webgreen,
    %pagecolor=RoyalBlue,
    %urlcolor=Black, linkcolor=Black, citecolor=Black, %pagecolor=Black,
    pdftitle={\myTitle},
    pdfauthor={\textcopyright\ \myName, \myUni, \myFaculty},
    pdfsubject={},
    pdfkeywords={},
    pdfcreator={pdfLaTeX},
    pdfproducer={LaTeX}
}

%**************************************************************
% Impostazioni di itemize
%**************************************************************
\renewcommand{\labelitemi}{$\bullet$}

%\renewcommand{\labelitemi}{$\bullet$}
%\renewcommand{\labelitemii}{$\cdot$}
%\renewcommand{\labelitemiii}{$\diamond$}
%\renewcommand{\labelitemiv}{$\ast$}


%**************************************************************
% Impostazioni di listings
%**************************************************************
\lstset{
    language=[LaTeX]Tex,%C++,
    keywordstyle=\color{RoyalBlue}, %\bfseries,
    basicstyle=\small\ttfamily,
    %identifierstyle=\color{NavyBlue},
    commentstyle=\color{Green}\ttfamily,
    stringstyle=\rmfamily,
    numbers=none, %left,%
    numberstyle=\scriptsize, %\tiny
    stepnumber=5,
    numbersep=8pt,
    showstringspaces=false,
    breaklines=true,
    frameround=ftff,
    frame=single
}


%**************************************************************
% Impostazioni di xcolor
%**************************************************************
\definecolor{webgreen}{rgb}{0,.5,0}
\definecolor{webbrown}{rgb}{.6,0,0}


%**************************************************************
% Altro
%**************************************************************

\newcommand{\omissis}{[\dots\negthinspace]} % produce [...]

% eccezioni all'algoritmo di sillabazione
\hyphenation
{
    ma-cro-istru-zio-ne
    gi-ral-din
}

\newcommand{\sectionname}{sezione}
\addto\captionsitalian{\renewcommand{\figurename}{Figura}
                       \renewcommand{\tablename}{Tabella}}

\newcommand{\glsfirstoccur}{\ap{{[g]}}}

\newcommand{\intro}[1]{\emph{\textsf{#1}}}

%**************************************************************
% Environment per ``rischi''
%**************************************************************
\newcounter{riskcounter}                % define a counter
\setcounter{riskcounter}{0}             % set the counter to some initial value

%%%% Parameters
% #1: Title
\newenvironment{risk}[1]{
    \refstepcounter{riskcounter}        % increment counter
    \par \noindent                      % start new paragraph
    \textbf{\arabic{riskcounter}. #1}   % display the title before the
                                        % content of the environment is displayed
}{
    \par\medskip
}

\newcommand{\riskname}{Rischio}

\newcommand{\riskdescription}[1]{\textbf{\\Descrizione:} #1.}

\newcommand{\risksolution}[1]{\textbf{\\Soluzione:} #1.}

%**************************************************************
% Environment per ``use case''
%**************************************************************
\newcounter{usecasecounter}             % define a counter
\setcounter{usecasecounter}{0}          % set the counter to some initial value

%%%% Parameters
% #1: ID
% #2: Nome
\newenvironment{usecase}[2]{
    \renewcommand{\theusecasecounter}{\usecasename #1}  % this is where the display of
                                                        % the counter is overwritten/modified
    \refstepcounter{usecasecounter}             % increment counter
    \vspace{10pt}
    \par \noindent                              % start new paragraph
    {\large \textbf{\usecasename #1: #2}}       % display the title before the
                                                % content of the environment is displayed
    \medskip
}{
    \medskip
}

\newcommand{\usecasename}{UC}

\newcommand{\usecaseactors}[1]{\textbf{\\Attori Principali:} #1. \vspace{4pt}}
\newcommand{\usecasepre}[1]{\textbf{\\Precondizioni:} #1. \vspace{4pt}}
\newcommand{\usecasedesc}[1]{\textbf{\\Descrizione:} #1. \vspace{4pt}}
\newcommand{\usecasepost}[1]{\textbf{\\Postcondizioni:} #1. \vspace{4pt}}
\newcommand{\usecasealt}[1]{\textbf{\\Scenario Alternativo:} #1. \vspace{4pt}}

%**************************************************************
% Environment per ``namespace description''
%**************************************************************

\newenvironment{namespacedesc}{
    \vspace{10pt}
    \par \noindent                              % start new paragraph
    \begin{description}
}{
    \end{description}
    \medskip
}

\newcommand{\classdesc}[2]{\item[\textbf{#1:}] #2}
