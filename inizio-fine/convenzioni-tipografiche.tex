% !TEX encoding = UTF-8
% !TEX TS-program = pdflatex
% !TEX root = ../relazione-finale.tex

%**************************************************************
% Convenzioni tipografiche
%**************************************************************
\phantomsection
\thispagestyle{empty}
\pdfbookmark{Struttura del documento e Convenzioni tipografiche}{Struttura del documento e Convenzioni tipografiche}

\chapter*{Struttura del documento}

Il presente documento è articolato in quattro capitoli:
\begin{description}
		\item[{\hyperref[cap:introduzione]{Il primo capitolo}}] presenta i temi su cui lo stage è stato svolto, elencando i rischi valutati per le scelte intraprese.

    \item[{\hyperref[cap:processi-metodologie]{Il secondo capitolo}}] descrive con maggior precisione il progetto di stage, elencandone gli obiettivi curricolari, formativi, tecnici e di prodotto.

    \item[{\hyperref[cap:descrizione-stage]{Il terzo capitolo}}] approfondisce il lavoro svolto durante lo svolgimento dello stage, esaminando le attività di analisi, progettazione, codifica e test.

    \item[{\hyperref[cap:analisi-requisiti]{Il quarto capitolo}}] fornisce una valutazione degli obiettivi raggiunti, motivando la presenza di eventuali obiettivi non raggiunti; inoltre vengono esaminate le conoscenze acquisite e i rischi descritti nel primo capitolo.

    % \item[{\hyperref[cap:progettazione-codifica]{Il quinto capitolo}}] approfondisce ...
    %
    % \item[{\hyperref[cap:verifica-validazione]{Il sesto capitolo}}] approfondisce ...
    %
    % \item[{\hyperref[cap:conclusioni]{Nel settimo capitolo}}] descrive ...
\end{description}

\let\clearpage\relax\chapter*{Convenzioni tipografiche}

Riguardo la stesura del testo, relativamente al documento sono state adottate le seguenti convenzioni tipografiche:
\begin{itemize}
	\item gli acronimi, le abbreviazioni e i termini ambigui o di uso non comune menzionati vengono definiti nel glossario, situato alla fine del presente documento;
	\item per la prima occorrenza dei termini riportati nel glossario viene utilizzata la seguente nomenclatura: parola\glsfirstoccur;
	\item i termini in lingua straniera o facenti parti del gergo tecnico sono evidenziati con il carattere \emph{corsivo}.
\end{itemize}
