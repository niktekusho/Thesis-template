% !TEX encoding = UTF-8
% !TEX TS-program = pdflatex
% !TEX root = ../relazione-finale.tex

%**************************************************************
% Convenzioni tipografiche
%**************************************************************
\phantomsection
\thispagestyle{empty}
\pdfbookmark{Struttura del documento e Convenzioni tipografiche}{Struttura del documento e Convenzioni tipografiche}

\chapter*{Struttura del documento}

Il presente documento è articolato in quattro capitoli:
\begin{description}
		\item[{\hyperref[cap:introduzione]{Nel primo capitolo}}] presento i temi su cui ho svolto lo stage, elencando i rischi che ho valutato per le scelte intraprese.

    \item[{\hyperref[cap:processi-metodologie]{Nel secondo capitolo}}] descrivo con maggior precisione il progetto di stage, elencandone gli obiettivi curricolari, formativi, tecnici e di prodotto.

    \item[{\hyperref[cap:descrizione-stage]{Nel terzo capitolo}}] approfondisco il lavoro svolto durante lo svolgimento dello stage, esaminando le attività di analisi, progettazione, codifica e test.

    \item[{\hyperref[cap:analisi-requisiti]{Nel quarto capitolo}}] fornisco una valutazione degli obiettivi raggiunti, motivando la presenza di eventuali obiettivi non raggiunti; inoltre esamino le conoscenze acquisite e i rischi descritti nel primo capitolo.

    % \item[{\hyperref[cap:progettazione-codifica]{Il quinto capitolo}}] approfondisce ...
    %
    % \item[{\hyperref[cap:verifica-validazione]{Il sesto capitolo}}] approfondisce ...
    %
    % \item[{\hyperref[cap:conclusioni]{Nel settimo capitolo}}] descrive ...
\end{description}

\let\clearpage\relax\chapter*{Convenzioni tipografiche}

Riguardo la stesura del testo, relativamente al documento ho adottato le seguenti convenzioni tipografiche:
\begin{itemize}
	\item ho definito un glossario, presente alla fine del presente documento, contenente gli acronimi, le abbreviazioni e i termini ambigui o di uso non comune menzionati nel corso del documento;
	\item ho utilizzato per la prima occorrenza dei termini riportati nel glossario la seguente nomenclatura: parola\glsfirstoccur;
	\item ho evidenziato i termini in lingua straniera o facenti parti del gergo tecnico con il carattere \emph{corsivo}.
\end{itemize}
