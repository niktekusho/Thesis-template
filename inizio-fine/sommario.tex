% !TEX encoding = UTF-8
% !TEX TS-program = pdflatex
% !TEX root = ../relazione-finale.tex

%**************************************************************
% Sommario
%**************************************************************
\cleardoublepage
\phantomsection
\pdfbookmark{Sommario}{Sommario}
\begingroup
\let\clearpage\relax
\let\cleardoublepage\relax
\let\cleardoublepage\relax

\chapter*{Sommario}

Il presente documento descrive il lavoro svolto dal laureando Nicola Dal Maso durante il periodo di stage interno individuale, della durata di circa trecento ore.
Lo scopo principale dello stage è coinciso con lo sviluppo e la realizzazione di un prototipo per la gestione dei dispositivi interconnessi (IoT) attraverso un'interfaccia \emph{web}.
Lo sviluppo della \emph{dashboard} è nato dall'esigenza di proporre una soluzione che unificasse in un unico centro di controllo tutti gli eventuali dispositivi connessi alla rete domestica dell'utente, permettendo tuttavia allo stesso di accedere all'interfaccia proprietaria di ciascun dispositivo.
Data la natura sperimentale del prodotto sviluppato, lo studente ha potuto scegliere un approccio architetturale innovativo per la realizzazione dell'applicazione: al fine di garantire scalabilità all'applicazione è stato richiesto allo studente lo studio dei princìpi fondamentali delle architetture a microservizi.


%\vfill
%
%\selectlanguage{english}
%\pdfbookmark{Abstract}{Abstract}
%\chapter*{Abstract}
%
%\selectlanguage{italian}

\endgroup			

\vfill

