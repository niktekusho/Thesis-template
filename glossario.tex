
%**************************************************************
% Acronimi
%**************************************************************
\renewcommand{\acronymname}{Acronimi e abbreviazioni}

\newacronym[description={\glslink{apig}{Application Program Interface}}]
    {api}{API}{Application Program Interface}

\newacronym[description={\glslink{umlg}{Unified Modeling Language}}]
    {uml}{UML}{Unified Modeling Language}

%**************************************************************
% Glossario
%**************************************************************
%\renewcommand{\glossaryname}{Glossario}

\newglossaryentry{apig}
{
    name=\glslink{api}{API},
    text=Application Program Interface,
    sort=api,
    description={in informatica con il termine \emph{Application Programming Interface API} (ing. interfaccia di programmazione di un'applicazione) si indica ogni insieme di procedure disponibili al programmatore, di solito raggruppate a formare un set di strumenti specifici per l'espletamento di un determinato compito all'interno di un certo programma. La finalità è ottenere un'astrazione, di solito tra l'hardware e il programmatore o tra software a basso e quello ad alto livello semplificando così il lavoro di programmazione}
}

\newglossaryentry{umlg}
{
    name=\glslink{uml}{UML},
    text=UML,
    sort=uml,
    description={in ingegneria del software \emph{UML, Unified Modeling Language} (ing. linguaggio di modellazione unificato) è un linguaggio di modellazione e specifica basato sul paradigma object-oriented. L'\emph{UML} svolge un'importantissima funzione di ``lingua franca'' nella comunità della progettazione e programmazione a oggetti. Gran parte della letteratura di settore usa tale linguaggio per descrivere soluzioni analitiche e progettuali in modo sintetico e comprensibile a un vasto pubblico}
}

\newglossaryentry{load balancerg}
{
  name=Load balancer,
  text=Load balancer,
  sort=load balancer,
  description={in informatica, il load balancer è lo strumento hardware o software attraverso cui viene distribuito un carico di lavoro su più risorse di elaborazione (\textit{load balancing})}
}

\newglossaryentry{throughputg}
{
  name=throughput,
  text=throughput,
  sort=throughput,
  description={}
}

\newglossaryentry{aspettativag}
{
  name=aspettativa,
  text=aspettativa,
  sort=aspettativa,
  description={è equivalente ad un periodo di astensione dal lavoro, previsto dalla legge, che il datore di lavoro può concedere ad un proprio lavoratore per motivi familiari o personali, generalmente non retribuito}
}

\newglossaryentry{kernelg}
{
  name=kernel,
  text=kernel,
  sort=kernel,
  description={è il sottosistema di un sistema operativo che fornisce ai processi in esecuzione sull'elaboratore l'accesso alle risorse fisiche, in modo controllato e sicuro}
}
