% !TEX encoding = UTF-8
% !TEX TS-program = pdflatex
% !TEX root = ../relazione-finale.tex

%**************************************************************
\chapter{Introduzione}
\label{cap:introduzione}
%**************************************************************

\intro{Nelle sezioni di questo capitolo parlerò dell'idea alla base dello svolgimento dello stage e dei rischi derivati dalle scelte effettuate.}\\

% \noindent Esempio di utilizzo di un termine nel glossario \\
% \gls{api}. \\
% 
% \noindent Esempio di citazione in linea \\
% \cite{site:agile-manifesto}. \\
% 
% \noindent Esempio di citazione nel pie' di pagina \\
% citazione\footcite{womak:lean-thinking} \\

%**************************************************************
\section{L'idea}

L'idea che mi ha spinto allo sviluppo e alla realizzazione del progetto di stage è stata quella unificare in un unico centro di controllo
tutti gli eventuali dispositivi connessi alla rete domestica dell'utente, permettendo 
tuttavia allo stesso di accedere all'interfaccia proprietaria di ciascun dispositivo.
L'obiettivo di una tale dashboard non è quindi \textit{intrappolare} l'utente in un unico ecosistema domotico, 
bensì quello di facilitare la consultazione delle informazioni più frequentemente richieste dall'utente provenienti da più ecosistemi distinti.

Per assicurare prestazioni, affidabilità (\textit{availability}) e scalabilità del prodotto sviluppato e per aggiungere una natura formativa al progetto di stage,
ho scelto di progettare il prodotto secondo l'architettura a microservizi.

\subsection{IoT}

Illustrare cosa si intende per IoT (fornendo un background informativo) e perchè è stato scelto questo tema

\subsection{Architettura a microservizi}

Illustrare cosa è l'Architettura a microservizi (fornendo un background informativo) e perchè è stato scelto questo tema.

%**************************************************************
\section{Rischi}

Illustrare i rischi derivati dalle scelte effettuate.

%**************************************************************
\section{Organizzazione del testo}

\begin{description}
    \item[{\hyperref[cap:processi-metodologie]{Il secondo capitolo}}] descrive ...

    \item[{\hyperref[cap:descrizione-stage]{Il terzo capitolo}}] approfondisce ...

    \item[{\hyperref[cap:analisi-requisiti]{Il quarto capitolo}}] approfondisce ...

    \item[{\hyperref[cap:progettazione-codifica]{Il quinto capitolo}}] approfondisce ...

    \item[{\hyperref[cap:verifica-validazione]{Il sesto capitolo}}] approfondisce ...

    \item[{\hyperref[cap:conclusioni]{Nel settimo capitolo}}] descrive ...
\end{description}

Riguardo la stesura del testo, relativamente al documento sono state adottate le seguenti convenzioni tipografiche:
\begin{itemize}
	\item gli acronimi, le abbreviazioni e i termini ambigui o di uso non comune menzionati vengono definiti nel glossario, situato alla fine del presente documento;
	\item per la prima occorrenza dei termini riportati nel glossario viene utilizzata la seguente nomenclatura: \emph{parola}\glsfirstoccur;
	\item i termini in lingua straniera o facenti parti del gergo tecnico sono evidenziati con il carattere \emph{corsivo}.
\end{itemize}
