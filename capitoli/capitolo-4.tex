% !TEX encoding = UTF-8
% !TEX TS-program = pdflatex
% !TEX root = ../relazione-finale.tex

%**************************************************************
\pagebreak
\chapter{Valutazione retrospettiva}
\label{cap:analisi-requisiti}
%**************************************************************

\section{Valutazione raggiungimento degli obiettivi}

Per valutare correttamente il raggiungimento degli obiettivi è necessario che riprenda gli obiettivi fissati nel capitolo ~\ref{cap:processi-metodologie}.
In particolare voglio osservare le tabelle in cui ho stabilito gli obiettivi di prodotto (~\ref{tab:obiettivi-prodotto}), gli obiettivi formativi (~\ref{tab:obiettivi-formativi}) e tecnici (~\ref{tab:obiettivi-tecnici}) del progetto.
Iniziando dalla valutazione degli obiettivi di prodotto, posso definirmi soddisfatto delle funzionalità implementate nel prototipo.
Ho soddisfatto gli obiettivi "OP2" e "OP3", a cui avevo assegnato importanza critica, e quindi il prototipo mostra correttamente quello che mi aspettavo di presentare.
L'utente della \emph{dashboard} del prototipo può visualizzare correttamente la lista di dispositivi collegati al sistema e, per ciascuno di essi, può vedere:
\begin{itemize}
  \item le specifiche generali che comprendono:
  \begin{itemize}
    \item nome del dispositivo;
    \item produttore del dispositivo;
    \item revisione del dispositivo;
  \end{itemize}
  \item le ultime misurazioni provenienti dal dispositivo, nell'unità di misura scelta dall'utente;
  \item se il dispositivo presenta funzionalità \emph{smart}, elementi grafici che permettono all'utente di avviare tali funzionalità.
\end{itemize}
Ho parzialmente implementato le funzionalità descritte negli obiettivi "OP1" e "OP5".
Per quanto riguarda "OP1", nella pagina principale della \emph{dashboard} l'utente può osservare lo stato dei servizi che compongono il sistema, tuttavia considero questo obiettivo solamente parzialmente soddisfatto perché mi aspettavo di riuscire a visualizzare maggiori informazioni, come ad esempio statistiche del traffico dati all'interno della rete.
Per quanto riguarda "OP5", l'utente può accedere a una pagina in cui può solamente cambiare due opzioni:
\begin{itemize}
  \item il proprio nome nel sistema;
  \item l'unità di misura della temperatura, utile nella visualizzazione delle temperature rilevate dai dispositivi legati a questo ambito.
\end{itemize}

Come ho già spiegato in ~\ref{val-req}, ho avuto difficoltà nelle attività di implementazione finali che hanno

Per ogni obiettivo definito precedentemente valuterei il suo soddisfacimento e descriverei le problematiche rilevate durante lo svolgimento dello stage.

\begin{table}[H]
\caption{Tabella di valutazione delle funzionalità implementate nel prototipo}
\label{tab:esito-obiettivi-prodotto}
\begin{tabularx}{\linewidth}{|c|X|c|c|}
\hline
\textbf{Id obiettivo} & \textbf{Descrizione obiettivo} & \textbf{Importanza funzionalità} & \textbf{Esito}\\
\hline
\label{OP1} OP1 & Visualizzare lo stato generale del sistema. & 4 & Parzialmente soddisfatto \\
\hline
\label{OP2} OP2 & Visualizzare quali dispositivi sono collegati al sistema. & 5 & Soddisfatto \\
\hline
\label{OP3} OP3 & Visualizzare le informazioni trasmesse dai dispositivi collegati al sistema. & 5 & Soddisfatto \\
\hline
\label{OP4} OP4 & Implementare sistemi di autenticazione dell'utente. & 2 & Omesso \\
\hline
\label{OP5} OP5 & Visualizzare e gestire le preferenze dell'utente. & 3 & Parzialmente soddisfatto \\
\hline
\end{tabularx}
\end{table}

\begin{table}[H]
\caption{Tabella di valutazione degli obiettivi formativi del progetto}
\label{tab:esito-obiettivi-formativi}
\begin{tabularx}{\linewidth}{|c|X|c|}
\hline
\textbf{Id obiettivo} & \textbf{Descrizione obiettivo} & \textbf{Esito}\\
\hline
\label{OF1} OF1 & Apprendere abilità elementari per la comprensione delle funzionalità richieste dal mercato IoT, specialmente nel campo della automazione domestica. \\
\hline
\label{OF2} OF2 & Acquisire conoscenze adeguate alla scelta e implementazione di un protocollo di comunicazione adeguato al campo di utilizzo del progetto. \\
\hline
\label{OF3} OF3 & Comprendere il concetto di architettura a microservizi, con i pregi e i difetti caratteristici di una tale architettura. \\
\hline
\label{OF4} OF4 & Acquisire le nozioni legate alla containerizzazione di un sistema \emph{software} in un contesto architetturale basato su microservizi. \\
\hline
\end{tabularx}
\end{table}

\begin{table}[H]
\caption{Tabella di valutazione degli obiettivi tecnici del progetto}
\label{tab:esito-obiettivi-tecnici}
\begin{tabularx}{\linewidth}{|c|X|c|}
\hline
\textbf{Id obiettivo} & \textbf{Descrizione obiettivo} & \textbf{Esito}\\
\hline
\label{OT1} OT1 & Rilascio del codice sorgente del prototipo e della documentazione associata nei termini di una licenza \gls{open source}. \\
\hline
\label{OT2} OT2 & La documentazione associata al progetto deve includere le istruzioni necessarie all'esecuzione del prototipo. \\
\hline
\label{OT3} OT3 & Il prototipo deve essere eseguibile su dispositivi presenti in una rete, previa corretta configurazione. \\
\hline
\label{OT4} OT4 & Il prototipo deve essere eseguibile su un dispositivo di test, che simuli l'esecuzione in un ambiente reale. \\
\hline
\label{OT5} OT5 & Il prototipo deve prevedere strumenti per gestire la scalabilità del sistema e per monitorarne lo stato. \\
\hline
\label{OT6} OT6 & Il prototipo deve essere implementato in \href{https://nodejs.org/en/about/}{Node.js} (\url{https://nodejs.org/en/about/}) per il lato server. \\
\hline
\label{OT7} OT7 & L'interfaccia utente del prototipo deve essere implementata in \href{https://reactjs.org/}{React} (\url{https://reactjs.org/}). \\
\hline
\end{tabularx}
\end{table}


%**************************************************************
\section{Conoscenze acquisite}

Autovalutazione delle conoscenze acquisite, ragionando in termini di aspettative iniziali e menzionando le parti che hanno causato difficoltà nello svolgimento del progetto.


%**************************************************************
\section{Conclusioni}
