% !TEX encoding = UTF-8
% !TEX TS-program = pdflatex
% !TEX root = ../relazione-finale.tex

%**************************************************************
\pagebreak
\chapter{Progetto di Stage}
\label{cap:processi-metodologie}
%**************************************************************

\intro{Nelle sezioni di questo capitolo parlerò delle aspettative e degli obiettivi posti inizialmente per il progetto di Stage}\\

%**************************************************************
\section{Obiettivi di prodotto}

Gli obiettivi di prodotto consistono in quelle caratteristiche e funzionalità importanti per gli utenti del sistema. \\

L'obiettivo principale del prodotto è quello di fornire un'interfaccia unificata per la gestione dei dispositivi connessi, consentendo all'utente l'accesso all'interfaccia proprietaria di ciascun dispositivo. \\
Date le limitate risorse (di tempo, di persone e di conoscenza pregressa) non mi aspetto che il prodotto sia lo stato dell'arte in questo settore, soprattutto per quanto riguarda l'ambito della sicurezza; allo stesso modo non è stata mia intenzione sviluppare un nuovo protocollo di comunicazione tra dispositivi, bensì mi sono limitato a studiare le possibili scelte e a motivare la particolare scelta del protocollo che utilizzato. \\

Come riporta il titolo della presente relazione, per il progetto di stage ho preferito concentrarmi su funzionalità per certi aspetti innovative restando nell'ambito di sviluppo di un prototipo; come tale, il prototipo non è una soluzione pronta alla distribuzione sul mercato (\emph{production-ready}), quanto un modo per sperimentare con l'automazione domestica introducendo funzionalità non diffuse nei prodotti presenti sul mercato. \\

Un secondo nodo cruciale su cui ho puntato sin dalle prime attività di sviluppo è la scalabilità della soluzione rispetto all'obiettivo principale citato precedentemente.
Ho concepito l'obiettivo di scalabilità del prodotto come una possibilità di decentralizzare le funzionalità che il sistema fornisce; l'idea, a mio modo di vedere innovativa, consiste nel realizzare un sistema \emph{software} che possa essere eseguito su dispositivi diversi e che possa gestire funzionalità "dinamiche" per i dispositivi con cui l'utente può interagire.
Un esempio che mi ha aiutato a chiarire questo obiettivo è il seguente: supponiamo di voler realizzare un sistema che gestisca l'automazione domestica per quanto riguarda l'illuminazione e per quanto riguarda la gestione termica dell'abitazione.

I dispositivi per l'illuminazione domestica consistono solamente in varianti di un solo dispositivo, la lampada. \\
I dispositivi per la gestione termica dell'abitazione sono invece sostanzialmente due:
\begin{itemize}
  \item sensori, i quali devono inviare informazioni relative alla temperatura di un determinato ambiente;
  \item termostati, i quali devono interfacciarsi con i sensori da una parte, per ricevere e sintetizzare la distribuzione termica dell'abitazione, e con l'impianto di riscaldamento dall'altra, per applicare le variazioni di temperatura richieste dall'utente.
\end{itemize}
\\
Ho concepito la scalabilità del prodotto al fine di permettere che le funzionalità di gestione dei dispositivi legati alla temperatura possano essere responsabilità di uno o più dispositivi che eseguano \emph{software} dedicato allo scopo.
Proseguendo l'esempio iniziato precedentemente, la componente del sistema legata alla temperatura potrebbe essere eseguita su un termostato dotato di capacità di elaborazione adeguate, oppure potrebbe essere eseguita su un dispositivo \emph{ad-hoc} oppure ancora su un dispositivo utilizzato in comune con gli apparati di illuminazione senza che le due componenti entrino in conflitto.
Lo stesso principio è applicabile alla gestione dell'illuminazione (lampada \emph{master} con elettronica in grado di eseguire il \emph{software}, dispositivo dedicato o dispositivo comune).


%**************************************************************
\section{Obiettivi curricolari}

Gli obiettivi curricolari sono quelle caratteristiche dimostrabili che sono rilevanti sul mercato del lavoro. \\

% Gli obiettivi curricolari su cui ho puntato per l'aspetto IoT servono a dimostrare le soluzioni che svilupperò .
Dal momento che il prodotto sviluppato per il progetto di stage è completamente \emph{open source} e rilasciato con la licenza \gls{MIT}, le aziende interessate ad alcuni aspetti del prodotto possono visualizzare, migliorare ed eventualmente contribuire al codice esistente, tenendo bene in considerazione che il prodotto è sviluppato da un solo sviluppatore al primo approccio con il tema. \\

Gli obiettivi curricolari relativi al paradigma dell'architettura a microservizi invece dimostrano la mia volontà di sperimentare con approcci non convenzionali allo sviluppo di sistemi \emph{software} che hanno particolari esigenze di scalabilità. \\
A mio avviso, il mercato del lavoro dovrebbe apprezzare l'esperienza fatta in questo ambito grazie al fatto che la documentazione del prodotto e il codice sorgente del prodotto stesso, essendo visualizzabili liberamente da tutti gli attori interessati, permettono loro di capire secondo quali princìpi ho sviluppato le componenti del sistema. \\
In particolare, le aziende interessate possono valutare le mie effettive competenze nello sviluppo di architetture distribuite e nell'implementazione di un protocollo di comunicazione esistente.

% TODO: unire meglio i due argomenti

Gli obiettivi formativi indicano le conoscenze e le abilità che dovrebbero essere studiate e imparate durante lo svolgimento dello stage.

Anche in questo caso, separo gli obiettivi formativi legati all'ambito IoT da quelli relativi al tema architettura a microservizi.

Dal punto di vista del tema IoT, ho studiato le caratteristiche essenziali e le funzionalità che i dispositivi IoT esistenti possiedono per permetterne una corretta rappresentazione all'interno del sistema.
Inoltre lo studio dei protocolli di comunicazione esistenti è stato fondamentale per permettere lo sviluppo di un sistema affidabile e performante. In particolare ho analizzato protocolli di comunicazione esistenti, utilizzabili liberamente e le cui specifiche siano accessibili pubblicamente e gratuitamente.
Anche se il prodotto sviluppato è un prototipo in cui la sicurezza non è ben implementata, ho studiato e valutato l'aspetto relativo alla sicurezza dei protocolli utilizzabili, considerando la diffusione del protocollo, le effettive vulnerabilità note e il rapporto sicurezza del protocollo rispetto alla sua facilità di utilizzo.

A mio modo di vedere, il lato formativo relativo all'architettura a microservizi è stato quello più impegnativo.
Essendo alla mia prima esperienza con la progettazione di un sistema distribuito e con un paradigma architetturale relativamente nuovo, mi aspetto di dover impiegare un tempo maggiore, rispetto alla parte relativa all'IoT, per imparare i concetti che permettono lo sviluppo delle architetture a microservizi.
In particolare, uno degli aspetti su cui mi sono concentrato con maggior attenzione è l'analisi della corretta dimensione di un servizio tale per cui esso possa essere definito "\emph{micro}".
Un altro concetto importante su cui ho dovuto informarmi con attenzione consiste nella scelta delle tecnologie di persistenza dei dati per ciascun servizio, specialmente in relazione alla già citata \emph{Polyglot persistence} (riferimento ~\ref{par:microservizi-intro}).
Mi aspetto quindi di dover imparare ad analizzare la struttura delle informazioni che ciascun dispositivo deve fornire al fine di sviluppare una componente che riesca a memorizzare ed elaborare tali informazioni in efficienza.


%**************************************************************
\section{Obiettivi tecnici}

Gli obiettivi tecnici consistono nelle tecniche e nelle tecnologie specifiche che andrò ad utilizzare durante lo svolgimento dello stage.

Il primo obiettivo tecnico su cui ho puntato è quello di distribuire l'insieme delle funzionalità del prodotto come \emph{software open source}: esso è pubblicato con una licenza permissiva che permette a chi è interessato di visualizzare, modificare ed utilizzare il prodotto senza condizioni vincolanti.

Dal momento che il prodotto è testabile da un pubblico potenzialmente vasto, il secondo obiettivo tecnico generale consiste nella distribuzione delle istruzioni per provare il sistema prodotto in una macchina locale.

Da un punto di vista tecnico, per l'ambito IoT mi aspetto di dover padroneggiare un protocollo di comunicazione, scelto tra quelli in esame, per permettere ai dispositivi di comunicare tra loro nelle modalità consentite dal protocollo.
I dispositivi dovranno comunicare tra loro sia con comunicazioni uno a uno, sia con comunicazioni uno a molti (\gls{broadcast}) e gestire eventuali errori di trasmissione in base all'importanza dell'informazione che essi trasmettono.

Per permettere a chiunque di provare il sistema, mi aspetto di dover fornire un modo per simulare la presenza di dispositivi IoT nella propria rete domestica.

Nell'ambito delle architetture a microservizi ho dovuto padroneggiare alcuni dei \gls{pattern} peculiari di queste architetture \emph{software}, come per esempio il \gls{Gatewaypattern}, e ho dovuto studiare le tecniche e le tecnologie che si sono diffuse per monitorare lo stato del sistema, per orchestrare i servizi e per gestire la scalabilità degli stessi.
