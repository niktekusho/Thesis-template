% !TEX encoding = UTF-8
% !TEX TS-program = pdflatex
% !TEX root = ../relazione-finale.tex

%**************************************************************
\chapter{Progetto di Stage}
\label{cap:processi-metodologie}
%**************************************************************

\intro{Nelle sezioni di questo capitolo parlerò delle aspettative e degli obiettivi posti inizialmente per il progetto di Stage}\\

%**************************************************************
\section{Obiettivi curricolari}

Questa sezione risponde alla domanda:

Quali valori mi \textbf{aspetto} di dedurre dall'esperienza di stage che siano rilevanti sul mercato del lavoro?

Parlerò della rilevanza del mercato IoT, in continua espansione.

Parlerò inoltre del crescente interesse delle aziende verso il paradigma dell'architettura a microservizi.


%**************************************************************
\section{Obiettivi formativi}

Questa sezione risponde alla domanda:

Quali informazioni mi \textbf{aspetto} di imparare/studiare durante lo svolgimento dello Stage?

Parlerò dei concetti imparati per supportare al meglio lo sviluppo di applicazioni con architettura a microservizi.
I concetti principali sono: polyglot persistence, gateway pattern, analisi della corretta dimensione di un servizio per essere definito un microservizio.
Per quanto riguarda l'IoT parlerei dei protocolli analizzati durante l'attività di formazione, soffermandomi in particolare su MQTT.


%**************************************************************
\section{Obiettivi tecnici}

Questa sezione risponde alle domande:

Quali tecnologie mi \textbf{aspetto} di imparare a padroneggiare durante lo svolgimento dello Stage?
Quali competenze mi \textbf{aspetto} di apprendere durante lo svolgimento dello Stage?

Per quanto riguarda l'architettura a microservizi, mi aspetto di padroneggiare l'utilizzo del gateway pattern e di cominciare a costruire esperienza sulla composizione di servizi e sul loro dimensionamento.
Per quanto riguarda l'IoT mi aspetto di aver appreso MQTT nel suo utilizzo come protocollo molti-a-molti, uno-a-molti e uno-a-uno.

%**************************************************************
\section{Obiettivi di prodotto}

Questa sezione risponde alla domanda:

Quali caratteristiche (innovative o non) del prodotto sviluppato mi \textbf{aspetto} di poter dimostrare?

L'aspetto innovativo del prodotto che mi aspetto di evidenziare è quello della scalabilità del prodotto e la possibilità di gestire dispositivi che dichiarino azioni.
