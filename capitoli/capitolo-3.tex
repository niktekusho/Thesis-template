% !TEX encoding = UTF-8
% !TEX TS-program = pdflatex
% !TEX root = ../relazione-finale.tex

%**************************************************************
\chapter{Svolgimento dello stage}
\label{cap:descrizione-stage}
%**************************************************************

\intro{Nelle sezioni di questo capitolo parlerò dell'effettivo svolgimento dello stage: organizzazione dello
stage, analisi dei requisiti, progettazione ad alto livello, documentazione prodotta, test sviluppati e
validazione dei requisiti.}\\

%**************************************************************
\section{Organizzazione dello Stage}

Qui posso parlare del tempo impegnato nello svolgimento dello stage e delle milestone raggiunte.

%**************************************************************
\section{Ambiente di sviluppo}

\subsection{Strumenti di sviluppo}

% Durante la fase di analisi iniziale sono stati individuati alcuni possibili rischi a cui si potrà andare incontro.
% Si è quindi proceduto a elaborare delle possibili soluzioni per far fronte a tali rischi.\\
%
% \begin{risk}{Performance del simulatore hardware}
%     \riskdescription{le performance del simulatore hardware e la comunicazione con questo potrebbero risultare lenti o non abbastanza buoni da causare il fallimento dei test}
%     \risksolution{coinvolgimento del responsabile a capo del progetto relativo il simulatore hardware}
%     \label{risk:hardware-simulator}
% \end{risk}

%**************************************************************
\section{Piano di Lavoro}

Approfondire il Piano di Lavoro prodotto.

%**************************************************************
\section{Analisi dei Requisiti}

Approfondire l'Analisi dei Requisiti prodotta.

%**************************************************************
\section{Progettazione}

Approfondire la Specifica Tecnica prodotta.


%**************************************************************
\section{Documentazione}

Come è stata prodotta la documentazione? A chi è rivolta? Come è documentato il codice?

%**************************************************************
\section{Test}

Come sono stati svolti i test? Coverage?


%**************************************************************
\section{Validazione dei Requisiti}
